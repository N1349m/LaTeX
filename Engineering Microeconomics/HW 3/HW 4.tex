\documentclass{article}

\usepackage[utf8]{inputenc}
\usepackage{graphicx}
\usepackage{amsmath}
\usepackage[letterpaper, portrait, margin=1in]{geometry}
\usepackage{booktabs}

\title{Engineering Microenconomics HW 4}
\author{Nikhil Menon}
\date{September 28, 2016}

\begin{document}
\maketitle

1a. $\$100*1.1=\$110$

1b. $\$110/1.1=\$100$

1c. $\$52.50/1.05=\$50$

1d. $\$121/1.1^2=\$100$

2. The present value of the costs is \$300. The PV of the revenues is $\$363/1.1^2=\$300$ at 10\% interest and $\$363/1.12^2=\$289.38$. Therefore, the PV would be \$0 at 10\% and -\$10.62 at 12\%. He should not enter the business at 12\% interest.

3. The PV of the costs are $\$3$ million and $\$5/1.1=\$4.54$ million, for a total of \$7.54 million. The present value of the benefits three years from now is 
$$\frac{(200000+50000)}{\frac{.1*1.1^{50}}{1.1^{50}-1}}=2.48 million$$

Therefore the PV of the benefits now is
$$\frac{2.48}{1.1^2}=2.05 million$$
Therefore, the net present value will be -\$5.49 million.

4. The present value of the costs for the road is $10/1.1^3=7.51$ million. The PV of the benefits 4 years from now will be
$$\frac{(100000)}{\frac{.1*1.1^{49}}{1.1^{49}-1}}=0.99 million$$
Then, the present value now is
$$\frac{0.99}{1.1^3}=0.74million$$
In total, the present value is -\$6.77 million.

For the campsites, the costs are the previously calculated PV and $2/1.1^4=1.37$ million for a total of \$8.88 million.
The PV of the benefits in 5 years is
$$\frac{(300000)}{\frac{.1*1.1^{48}}{1.1^{48}-1}}=2.97 million$$
Then, the present value now is
$$\frac{2.96}{1.1^4}=2.03million$$
Added to the previous benefits, total benefits is \$2.77 million the present value is -\$6.11 million.

The company should neither build the roads nor build the roads and the campsite.

5. The PV of the cost would be $3/1.1^52=\$21000$ regardless of the situation. 

If there was just a dam, the PV of the benefits 52 years from now would be
$$\frac{(200000+50000)}{\frac{.1*1.1^{50}}{1.1^{50}-1}}=2.48 million$$
$$\frac{2.48}{1.1^{51}}=\$19000$$

Therefore, the present value is $-2000$.

If there was a road as well, the PV of the benefits 52 years from now would be 
$$\frac{(100000)}{\frac{.1*1.1^{50}}{1.1^{50}-1}}=0.99 million$$
$$\frac{0.99}{1.1^{51}}=\$8000$$

Added to the previously calculated benefits, the new PV is \$6000.

If there was a road as well, the PV of the benefits 52 years from now would be 
$$\frac{(100000)}{\frac{.1*1.1^{50}}{1.1^{50}-1}}=2.97 million$$
$$\frac{2.97}{1.1^{51}}=\$23000$$

Added to the previously calculated benefits, the new PV is \$29000.

The company should do maintenance if they have a road or if they have both a road and a campsite at the dam.

6.	A cost-effectiveness analysis would be useful when buying food. For example, you want to go buy cereal, but there are many different brands. In this case, you could measure the taste on a scale of 1 to 10 and compare that to the cost to create a cost-effective ratio. 

Doing a cost-benefit analysis would be useful when deciding between going out and buying food vs. cooking it yourself. Here, you would value some of the same things as before along with the price of the ingredients you have and your reservation price for cooking. If you really dislike cooking or don't have ingredients, the marginal benefits would be greater for going out than the marginal costs.

\end{document}