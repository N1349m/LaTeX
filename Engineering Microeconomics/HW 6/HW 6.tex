\documentclass{article}

\usepackage[utf8]{inputenc}
\usepackage{graphicx}
\usepackage{amsmath}
\usepackage[letterpaper, portrait, margin=1in]{geometry}
\usepackage{booktabs}

\title{Engineering Microenconomics HW 6}
\author{Nikhil Menon}
\date{November 28, 2016}

\newcommand{\makefig}[2]{
\begin{figure}[h]
\centering
\includegraphics[scale=#1]{#2}
\end{figure}
}

\begin{document}
\maketitle

1. B

2. Every person decides between leaving early or leaving on-time. To each person, leaving early is the better option. However, since a large portion of workers decide to leave early, the rush hour is effectively moved back. There are also people who decide to leave at the regular time, which just pushes the rush hour to be even longer. Thus, the rush hour now extends from the beginning of the early time to the end of the normal time.

3a.

3b. The equilibrium outcome is that everyone studies a lot. From the students' perspective, this is not the best outcome. To them, the best outcome would be if they studied a lot and he studied a little. 

4. Neither player has a dominant strategy

5a. He would not open the office. The earnings for opening +dishonest is worse than not opening at all

5b. He would open it. 

6a. She would order neither of the deserts. The reservation cost is less than the actual cost.

6b. She would order the chongos. The economic surplus fro the chongos is \$3 and the economic surplus for the flan is \$2.4.

7.

8. 

9. C

10. C

13. D


 
\end{document}