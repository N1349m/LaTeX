\documentclass{article}

\usepackage[utf8]{inputenc}
\usepackage{graphicx}
\usepackage{amsmath}
\usepackage[letterpaper, portrait, margin=1in]{geometry}
\usepackage{booktabs}

\title{Engineering Microenconomics HW 6}
\author{Nikhil Menon}
\date{November 28, 2016}

\newcommand{\makefig}[2]{
\begin{figure}[h!]
\centering
\includegraphics[scale=#1]{#2}
\end{figure}
}

\begin{document}
\maketitle

1. B

2. Every person decides between leaving early or leaving on-time. To each person, leaving early is the better option. However, since a large portion of workers decide to leave early, the rush hour is effectively moved back. There are also people who decide to leave at the regular time, which just pushes the rush hour to be even longer. Thus, the rush hour now extends from the beginning of the early time to the end of the normal time.

3a. \makefig{0.9}{P3.png}

3b. The equilibrium outcome is that everyone studies a lot. From the students' perspective, this is not the best outcome. To them, the best outcome would be if they studied a lot and he studied a little. 

4. Neither player has a dominant strategy

5. \makefig{0.9}{P5.png}

5a. He would not open the office. The manager would definitely be dishonest because that is economically the better option for him. The earnings for the owner for opening with a dishonest manager are worse than not opening at all.

5b. He would open it. Since the person would be willing to pay up to \$15000 a week to avoid dishonesty, the manager would choose not to cheat, and a functional shop is better than no shop for the owner.

6a. She would order neither of the deserts. The reservation cost is less than the actual cost.

6b. She would order the chongos. The economic surplus fro the chongos is \$3 and the economic surplus for the flan is \$2.4.

7. \makefig{0.8}{P7.png}

Both Swarupa and Samantha would choose to protect their forests.

8. D

\makefig{0.8}{P8.png} 

The tax would be \$65 for Swarupa and \$25 for Samantha. The final matrix is shown below

\makefig{0.8}{P8b.png}

In the end, the optimal choice is for both of them to protect their forests.
9. C

10. C

11. Because farmers can make more profit, land-owners can charge higher prices for using their land. Therefore, in the end, farmers will end up making the same amount of profit they did before.

12. Whenever a lane starts to move faster, people from slower lanes notice and move over to that lane, causing it to then slow back down. If a lane starts getting too slow, people start to move out of that lane to others, making that lane speed up again.

13. D


 
\end{document}