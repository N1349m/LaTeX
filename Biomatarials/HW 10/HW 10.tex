\documentclass{article}

\usepackage[utf8]{inputenc}
\usepackage{graphicx}
\usepackage{amsmath}
\usepackage[letterpaper, portrait, margin=1in]{geometry}
\usepackage{booktabs}

\title{Biomaterials HW 10}
\author{Nikhil Menon}
\date{December 2nd, 2016}

\newcommand{\makefig}[2]{
\begin{figure}[h]
\centering
\includegraphics[scale=#1]{#2}
\end{figure}
}

\begin{document}
\maketitle

\section{}
\subsection{}
This modification could be done using a plasma to radicalize the PS surface, creating hydroxyl or carboxyl radicals. These radicals would then react with the PEG chain through free radical polymerization.

\subsection{}
The surface polarity would be greater for the PEG modified surface than the pristine PS surface because PEG is a polar molecule, making the outermost surface is therefore polar. The PEG surface is also more wettable because a polar surface is hydrophilic, attracting water molecules. Therefore, the contact angle between the water drops and the surface would increase, which is a property of low wettability.

\subsection{}
For this, we could use a goniometer to measure the contact angle between water and the surface. As stated earlier, a larger contact angle means a lower wettability and a smaller contact angle means greater wettability.

\subsection{}
The critical surface tension is the surface tension at which a liquid completely wets a surface. Mathematically, this occurs when the cosine of the contact angle is 1. The critical surface tension for PS is 33 dynes/cm and for PEG is 43 dynes/cm. From these values, we can find the contact angles to be 63.1 degrees for PS and 53.9 degrees for  PEG. As shown by these numbers, PEG has better wettablity than a pristine PS surface. For the surface to be more wettable, the critical surface tension of the surface has to be above that of water.

\subsection{}
To characterize the composition of the surfaces, we could use ATR-IR and XPS. XPS measures the energy of the emitted electrons and can be used to measure the top 1-10 nm of the surface. The results here would show how many PEG strands are attached and extend outwards from the surface. ATR-IR uses a similar process but is able to measure to a greater depth, between 0.5 and 2 micrometers. This would be able to provide better detail about the composition closer to the PS surface, but it would also compromise by losing some detail at the ends of the PEG chains.

\subsection{}
Two examples are surface energy and surface topography. AFM and contact angle analysis could be used to measure these properties.

\section{}
\subsection{}
Attached on back

\subsection{}
For part (a), we could use XPS because the radicalized surface does not have anything attached to it. Then, to analyze parts (b) and (c), we could use XPS or ATR-IR, the methods for which are explained in  Problem 1. Either of these two methods would allow characterization, but ATR-IR would be preferred, since the surface thickness has increased.

\subsection{}
To measure topography, we could use AFM since the surface is non-conductive.

\section{}
\subsection{}
The driving force is the rearrangement of the protein surface to make the hydrophobic sections of the albumin bind to the PS surface.

\subsection{}
Some factors include the choice of solvent, the presence of other proteins, how polar the surface is, and the presence and interaction of functional groups on the surface with the albumin.

\subsection{}
The adsorption density would increase linearly with the concentration initially, but would eventually plateau at a certain level.
\makefig{0.7}{P3.png}

\subsection{}
The same factors would affect the surface adsorption as stated in question 2. In addition to this, the amount of albumin would also have a greater influence on the surface activity as shown below. The adsorption would go at a faster rate initially but would still reach the same plateau.
\makefig{0.7}{P3b.png}
To hydrophilize the surface, we could treat it with oxygen plasma.
\end{document}