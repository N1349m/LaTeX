\documentclass{article}

\usepackage[utf8]{inputenc}
\usepackage{graphicx}
\usepackage{amsmath}
\usepackage[letterpaper, portrait, margin=1in]{geometry}
\usepackage{booktabs}

\title{Biomaterials HW 4}
\author{Nikhil Menon}
\date{October 5th, 2016}

\begin{document}

\maketitle

1. We know that for a living anionic polymerization
$$M_n=\frac{[M]_0}{[I]_0}*M_0$$
where $[M]_0$ and $[I]0$ are the initial concentrations of 

$$[Na]=2*0.0033 mol = 0.0066 mol = 0.153 g$$

To completely stop the polymerization, all the molecules of sodium have to react with water, so there must be equal mols of sodium and water.
$$0.0066 mol * 18 \frac{g}{mol}=0.12 g$$

Even a small amount of water can completely terminate the reaction, so the polymerization should be conducted in a closed system with an inert gas.

2. As shown in the reaction, the polymer is a triblock where the length of the isoprene block is double the length of the styrene blocks.

We can calculate the mols of all the chemicals
$$[n-butyllithium]=0.2 mol/L*0.008 L =0.0016 mol$$

We also now that for an anionic living reaction,
$$M_n=\frac{[M]_0}{[I]_0}$$

We can then use this formula to calculate the number-average molecular weight for each block
$$M_n(styrene)=\frac{18 g}{0.0016 mol}=11250$$
$$M_n(isoprene)=\frac{60 g}{0.0016 mol}=37500$$
$$M_n=2*M_n(styrene)+2*M_n(isoprene)=97500$$
\end{document}