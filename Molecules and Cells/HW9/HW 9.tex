\documentclass{article}

\usepackage[utf8]{inputenc}
\usepackage{graphicx}
\usepackage{amsmath}
\usepackage[letterpaper, portrait, margin=1in]{geometry}
\usepackage{gensymb}

\title{Molecules and Cells HW 9}
\date{October 27th, 2016}

\newcommand{\makefig}[2]{
\begin{figure}[h]
\centering
\includegraphics[scale=#1]{#2}
\end{figure}
}

\begin{document}
\maketitle
1a. The rate would increase. ATP is an allosteric inhibitor, so if it is not able to bind the rate would increase.

1b. The rate would increase. citrate is an inhibitor, so if it is not able to bind the rate would increase.

1c. The rate would decrease. An allosteric activator can no longer bind, so the overall rate would decrease.

1d. The rate would increase. fructose 2,6-biphosphate is an allosteric activator in this reaction.

1e. The rate would decrease. At high concentrations, ATP acts as an allosteric inhibitor.

2a. The pmf would decrease. Since the membrane is now permeable, the hydrogen molecules would just diffuse to equalize the concentrations on both sides of the membrane, so the pmf would sharply decrease. 

2b. The temperature would increase since uncoupling oxidative oxidative phosphorylation cause an increase in heat. More sugar would have to be consumed to produce the same amount of ATP

2c. A person taking DNP could eat more without gaining weight since the found could not be converted into ATP due to the lack of a proton gradient. Since there is no ATP, the body can't make lipids or other forms of fats. However, people do need some ATP to live, so it was probably easy to go from losing weight to not making enough energy to survive.

2d. The pmf would decrease. As $K^+$ crosses the membrane, the membrane potential would decrease because the charge difference decreases, so there would be a smaller pmf.

3. E. An increase in ADP would increase the rate of glycolysis which would in turn increase the rate of oxidative phosphorylation.

4. Plugging in the given values into the provided equations gives that the $\Delta G$ available is -6.5 kcal/mol. Since the actual delta G needs to be 11 kcal/mol, 2 more protons would need to enter the cell through ATP synthase to balance this out.


5. Because the transport is coupled, the $\Delta G$ of the glucose transport has to be less than or equal to the $\Delta G$ for the Na transport. Using the Nernst equations for the two molecules gives
$$2.3RTlog(\frac{[Na^+]^2_{in}}{[Na^+]^2_{out}}+2FV\leq 2.3RTlog(\frac{[glucose]_{in}^2}{[glucose]_{out}}$$
Using this equation to solve for the ratio, which is $5.36 * 10^{-5}$

6. C

7. E

8. C

9. Looking at concentrations, $Na^+,Mg^{2+},Ca^{2+}$, and $Cl^-$ move inward, and $K^+$, and $H^+$ move outward. For all ions, they could be coupled with an ion of the same charge if going in opposite directions or with an ion of the same charge if moving in the same direction. Here, Na, Mg, and Ca could be co-transported with K or H. For Mg and Ca, they must take 2 of those two ions since they have a charge of 2. The missing anions are the bicarbonate anion, $HCO_3^-$, and maybe $HPO_4^-$

10. (I) With the electrochemical gradient. Driven by $\Delta pH$ 

(II) With the electrochemical gradient. Driven by membrane potential 

(III) With the electrochemical gradient. Driven by $\Delta pH$ 

(IV) Against the gradient

(V) Unaffected by the gradient

11. \makefig{0.5}{P11.png}

11c. The energy for $Na^+$ must be negative since both the energy from concentration and the energy from electric potential are negative. The sign of $\Delta G$ cannot be easily predicted for the other 2 ions.

12. Because A is being oxidized(losing electrons), its redox potential is the opposite of the provided value. Therefore, the overall redox potential is positive, so the $\Delta G$ is negative. The opposite reaction might occur with the correct application of electrical current.
13a. From the equations,
$$\Delta G=-nF\Delta E, \Delta E=\Delta E_0-\frac{2.3RT}{nF}log\frac{[lactate][NAD^+]}{[pyruvate][NADH]}$$
Since all reactants and products have a concentration of 1M, the log term is 0 and
$$\Delta G=-nF\Delta E_0=-5.98 kcal/mol$$

13b. The term inside the log becomes
$$\frac{[lactate]}{[pyruvate]}*\frac{[NAD^+]}{[NADH]}=1*1=1$$
So the Gibbs energy would be the same as before, -5.98 kcal/mol

13c. We can make
the equation
$$\frac{-\Delta G}{nF}=\Delta E_0-\frac{2.3RT}{nF}log\frac{[lactate][NAD^+]}{[pyruvate][NADH]}$$
which becomes
$$\frac{[lactate][NAD^+]}{[pyruvate][NADH]}=10^{(nf\Delta E_0+\Delta G)/2.3RT}=1.72*10^4$$

13d. Substituing all the values into the original equation
$$\Delta G=-nF(\Delta E_0-\frac{2.3RT}{nF}log\frac{[lactate][NAD^+]}{[pyruvate][NADH]})$$
gives that $\Delta G = -0.74 kcal/mol$

14a. The proton pumping would decrease because the gradient would not be as steep, so the pmf would be smaller

14b. Oxygen consumption would increase to offset the decrease in the gradient from the ionophore. 

14c. The ratio would decrease because ATP is being produced at a much slower rate due to the decrease in the electrochemical gradient.

14d. $$\Delta G = \Delta G_0 +2.3RTlog(\frac{[ATP]}{[ADP][P_i]})$$

14e. From the previous equation, the ratio of ATP to ADP would decrease, so the log quantity would decrease and $\Delta G$ would also decrease.

15a. This is wrong. A low pH increases the concentration of acid, not base.

15b. This is true. The $H^+$ ions aggregate in the inter-membrane space, and this is the primary force that drives ATP production 

15c. This is false. There is a high concentration of acid in the inter-membrane space, not in the matrix itself. An increase in the acid concentration in the matrix would actually decrease ATP production. 

15d. This is false. Decreasing the pH would not change the amount of $OH^-$ in the mitochondrial matrix.

\end{document}