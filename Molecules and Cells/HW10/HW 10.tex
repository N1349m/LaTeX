\documentclass{article}

\usepackage[utf8]{inputenc}
\usepackage{graphicx}
\usepackage{amsmath}
\usepackage[letterpaper, portrait, margin=1in]{geometry}
\usepackage{gensymb}

\title{Molecules and Cells HW 10}
\date{November 10th, 2016}

\newcommand{\makefig}[2]{
\begin{figure}[h]
\centering
\includegraphics[scale=#1]{#2}
\end{figure}
}

\begin{document}
\maketitle
1. This occurs because the extracellular concentration of $Na^+$ ions is so much greater than the other two ions. It is more likely for a sodium ion to reach the channel than either of the other two ions.

2. D. As the action potential reaches the end of the neuron (the synapse), the potential cause voltage-gated $Ca^{2+}$ channels to open, which then signals for the release of acetylcholine.

3i. This current indicates the the voltage-gated channel is open.

3ii. The increase in the peak could be due to an influx of $Ca^{2+}$ ions. Because they have double the charge of the sodium ions, the current would also approximately double in magnitude.

3iii. The spikes in the graph would not occur. Acetylcholine binds to the outside surface of the channel, so it would not be able to bind because the pipette would stop it from interacting with the channel.

4. More acetylcholine would need to be released to get a larger stimulus. The mutant channel would only activate at about -20 mV, while the regular channel would activate at about -40 mV.

5. The Nernst potential for $K^+$ is negative, the Nernst potential for $NA^+$ is positive, and the Nernst potential for $Cl^-$ is negative. For both sodium and potassium, the ions are positive, so the determinant of sign is the log term. If the inside of the log term is greater than 1, the potential is positive, and vice versa. For $Cl^-$, the ion is negative and the log term is positive, so the potential is negative.

5a.	More difficult. $Cl^-$ would flow inside, making the cell more negative. Therefore, it would take a larger stimulus to get to the threshold potential.

5b. Because $Na^+$ is the only ion whose potential positive, it has to have the greatest conductance to make the total potential positive.

6. To find the energy needed, we can use the equation
$$\Delta G = RTln(\frac{[N]_{out}}{[N]_{in}})+nFE$$
The voltage term should be negative, because a positive ion is moving from positive to negative, and the concentration term should be positive because the ion is moving from low concentration to high contration. Therefore,
$$\Delta G = -0.89$$

We can solve for the Nernst potential using the equation
$$\frac{61.4}{z}log(\frac{[Alien++]_{out}}{[Alien++]_{in}})=31.2$$
Because the Nernst potential is less than the membrane potential, ions would flow to the outside of the cell.

7. E
\end{document}