\documentclass{article}

\usepackage[utf8]{inputenc}
\usepackage{graphicx}
\usepackage{amsmath}
\usepackage[letterpaper, portrait, margin=1in]{geometry}
\usepackage{booktabs}
\usepackage{enumitem}

\title{Physical Chemistry HW 4}
\author{Nikhil Menon}
\date{March 6,2017}

\newcommand{\makefig}[2]{
\begin{figure}[h!]
\centering
\includegraphics[scale=#1]{#2}
\end{figure}
}

\newcommand{\ang}{\mbox{\AA}}

\begin{document}

\maketitle
\begin{enumerate}
\item

a. We can model the movement of the interstitial atom as a two-dimensional random walk. Using the equation for mean-square distance
$$\left<x^2\right>=(2d)\Gamma_{hop}t\alpha^2$$
Substituting in the given values gives
$$\left<x^2\right>=2dt\alpha^2(v_ie^{\frac{-(G_{A*}-G_{A})}{kT}})=2.05*10^7 \ang$$
Finally, the total distance traveled is
$$x=\sqrt{2.05*10^7 A} = 4547 \ang$$

b. Here, we use a 1-dimensional random walk with the same formula as above
$$\left<x^2\right>=2dt\alpha^2(v_ie^{\frac{-(G_{A*}-G_{A})}{kT}})=4492 \ang$$
The total distance traveled is
$$x=\sqrt{4492 \ang}=67.02 \ang$$

c. We can estimate the distance using Pythagorean theorem
$$d=\sqrt{67.02^2+4547^2}=4468.5 \ang$$

d. This is exactly what we expected. The distance should be slightly longer since the particle is able to diffuse in 3 dimensions, but no too much higher since the activation energy in the z direction is greater.

e. Using the given correlation factor, the new equation is
$$\left<x^2\right>=2fx_idt\alpha^2(v_ie^{\frac{-(G_{A*}-G_{A})}{kT}}) = 2878 \ang$$
The answer has decreased, which makes sense since now some portion of previously successful jumps are unsuccessful.

\item

a. Based on Vineyard's model,
$$\Gamma=ve^{\frac{-\Delta G_{A^*}}{kT}}$$
For the first,
$$\Gamma_1=8*10^12*e^{\frac{-0.34}{8.617*10^5*300 K}}=1.55*10^7 Hz$$
For the second,
$$\Gamma_2=8.57*10^7 Hz$$
From these values, we can calculate the mean square speed
$$\frac{\left<x^2\right>}{t}=2d\Gamma\alpha^2$$
For the first conductor,
$$\frac{\left<x^2\right>}{t}=2*3*1.55*10^7 Hz*(4.2 \ang)^2=1.64*10^9 \frac{\ang}{s}$$
For the second,
$$\frac{\left<x^2\right>}{t}=4.94*10^9 \frac{\ang}{s}$$
Thus, ions conduct faster in the second conductor.

b. Because diffusion in these materials is modeled with Vineyard's model, the vibrational frequency should not change with temperature.

c. At 600 K, the new jump frequencies are
$$\Gamma_1=1.11*10^{10} Hz$$
$$\Gamma_2=1.31*10^{10} Hz$$
Plugging these values into the formula for mean square speed gives
$$\frac{\left<x^2\right>}{t}=1.18*10^{12}  \frac{\ang}{s}$$
for the first conductor and
$$\frac{\left<x^2\right>}{t}=7.55*10^{11}  \frac{\ang}{s}$$
Thus, the first conductor now conducts lithium ions more quickly.

d. We can set up the equation
$$2dv_1e^{\frac{-\Delta G_{A^*_1}}{kT}}\alpha_1^2=2dv_2e^{\frac{-\Delta G_{A^*_2}}{kT}}\alpha_2^2$$
which simplifies to become
$$v_1e^{\frac{-\Delta G_{A^*_1}}{kT}}\alpha_1^2=v_2e^{\frac{-\Delta G_{A^*_2}}{kT}}\alpha_2^2$$

Solving for T gives
$$T=465.7 K$$

e. Setting the two new expressions for speed equal to each other gives
$$\alpha_A*\frac{k_b*T}{h}*e^{\frac{-\Delta_{A^*}}{k_bT}}=\alpha_B*\frac{k_b*T}{h}*e^{\frac{-\Delta_{A^*}}{k_bT}}\rightarrow\alpha_A*e^{\frac{-\Delta_{A^*}}{k_bT}}=\alpha_B*e^{\frac{-\Delta_{A^*}}{k_bT}}$$
Solving for T gives
$$T=1511.2 K$$
This answer is significantly higher than the answer for part d.
\end{enumerate}
\end{document}https://www.wolframalpha.com/