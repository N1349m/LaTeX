\documentclass{article}

\usepackage[utf8]{inputenc}
\usepackage{graphicx}
\usepackage{amsmath}
\usepackage[letterpaper, portrait, margin=1in]{geometry}
\usepackage{booktabs}
\usepackage{enumitem}

\title{Physical Chemistry HW 3}
\author{Nikhil Menon}
\date{February 27,2017}

\begin{document}

\maketitle
\begin{enumerate}
\item  
a. The equilibrium constant $K$ is equal to
$$\frac{k_1}{k_2}=\frac{[B][C]}{[A]}=\frac{1}{2}$$
We also know that 1 mol of A becomes 1 mol of A becomes 1 mol of B and 1 mol of C. If the amount of mol B at equilibrium is $x$,
$$\frac{1}{2}=\frac{x^2}{1-x}$$
Solving for x gives
$$x=0.5$$
Thus, at equilibrium, there would be 0.5 mol of A.

b. The decomposition of D creates B, which is a product in the decomposition of A. Thus, when D is added, the amount of A increases and the amount of C decreases.

c. The rate equation for B is
$$\frac{-d[B]}{dt}=k_2[B][C]-k_1[A]+k_4[B][E]-k_3[D]$$
which equals 0 at equilibrium. Thus,
$$0=k_2[B][C]-k_1[A]+k_4[B][E]-k_3[D]$$
Solving for [B] gives
$$[B]=\frac{k_1[A]+k_3[D]}{k_2[C]+k_4[E]}$$
\end{enumerate}
\end{document}