\documentclass{article}

\usepackage[utf8]{inputenc}
\usepackage{graphicx}
\usepackage{amsmath}
\usepackage[letterpaper, portrait, margin=1in]{geometry}
\usepackage{booktabs}
\usepackage{enumitem}

\title{Physical Chemistry HW 3}
\author{Nikhil Menon}
\date{February 27,2017}

\newcommand{\makefig}[2]{
\begin{figure}[h!]
\centering
\includegraphics[scale=#1]{#2}
\end{figure}
}

\begin{document}

\maketitle
\begin{enumerate}
\item  

a. The rate equation here is
$$\frac{d[A]}{dt}=-k_1[A]+k_2[B][C]$$
which has to equal 0 at equilibrium. We also know that $[A]+[B]=1$ and $[B]=[C]$.
Substituting this in gives
$$0=-2k_2[A]-k_2(1-[A])^2\rightarrow 2[A]=(1-[A])^2$$
Solving this for $[A]$ gives
$$[A]=0.26 M$$

b. The decomposition of D creates B, which is a product in the decomposition of A. Thus, when D is added, the amount of A increases and the amount of C decreases.

c. The rate equation for B is
$$\frac{-d[B]}{dt}=k_2[B][C]-k_1[A]+k_4[B][E]-k_3[D]$$
which equals 0 at equilibrium. Thus,
$$0=k_2[B][C]-k_1[A]+k_4[B][E]-k_3[D]$$
Solving for [B] gives
$$[B]=\frac{k_1[A]+k_3[D]}{k_2[C]+k_4[E]}$$

\item
The number of edge sites in the nanoparticle is
$$12N-16$$
and the number of face sites is 
$$6(N-2)^2$$
We also can express the total fraction of occupied sites as
$$\theta = x_{face}\theta_{face}+x_{edge}\theta_{edge}$$
where $x_{face}$ and $x_{edge}$ represent the probabilities of a space being a face or an edge, respectively.

We also know that
$$K_1=\frac{k_{adsorption}}{k_{desorption}},K_1*k_{desorption}=k_{adsorption}$$
for the edge sites, and
$$K_2=\frac{k_{adsorption}}{k_{desorption}},K_1*k_{desorption}=k_{adsorption}$$

First, we can solve for the respective probabilities
$$x_{face}=\frac{N_{face}}{N_{total}}=\frac{6(N-2)^2}{6(N-2)^2+12N-16}=\frac{3N^2-12N+12}{3N^2-6N+4}$$
$$x_{edge}=\frac{N_{edge}}{N_{total}}=\frac{12N-16}{6(N-2)^2+12N-16}=\frac{6N-8}{3N^2-6N+4}$$

Then, to find the two $\theta$ values, we can apply the langmuir equilibrium expression
$$\theta=\frac{k_{adsorption}C_M}{k_{desorption}+k_{adsorption}C_M}$$
Substituting in the previous expression with $K_1$ gives
$$\theta_{edge}=\frac{K_1k_{desorption}C_M}{k_{desorption}+K_1k_{desorption}C_M}=\frac{K_1C_{CO}}{1+K_1C_{CO}}$$
We can derive a similar expression for $\theta_{face}$
$$\theta_{face}=\frac{K_2C_{CO}}{1+K_2C_{CO}}$$
Finally, combining everything gives
$$\theta =\frac{K_1C_{CO}}{1+K_1C_{CO}}*\frac{6N-8}{3N^2-6N+4}+\frac{K_2C_{CO}}{1+K_2C_{CO}}*\frac{3N^2-12N+12}{3N^2-6N+4}$$

\item
\makefig{0.6}{P3.png}

\item
The rate equation here is
$$\frac{d[A]}{dt}=-k[A][C]$$
Solving the differential equation gives
$$[A]=[A]_0e^{-k[C]t}$$
At t=5, this becomes
$$[A]=e^{-0.1*5*2}=e^{-1}=0.367$$
Thus, the concentration of B would be
$$[B]=1-[A]=0.632$$

\end{enumerate}
\end{document}